% This is LLNCS.DEM the demonstration file of
% the LaTeX macro package from Springer-Verlag
% for Lecture Notes in Computer Science,
% version 2.4 for LaTeX2e as of 16. April 2010
%
\documentclass{llncs}
%
\usepackage{makeidx}  % allows for indexgeneration
\usepackage{booktabs}
\usepackage{todonotes}
\usepackage{microtype}
\usepackage{color}
%
\newcommand{\tnote}[3]{{\color{#2}#1: #3}}
\newcommand{\change}[1]{\textcolor{red}{#1}}
\newcommand{\DaH}[1]{\tnote{DaH}{red}{#1}}
\newcommand{\VKF}[1]{\tnote{VKF}{green}{#1}}
\newcommand{\JT}[1]{\tnote{JT}{blue}{#1}}
\newcommand{\IGNORE}[1]{}
%
\begin{document}
%
\title{Segmentation Error Suggestion for Semi-automatic EM Connectomics Proofreading}
%
\titlerunning{Learning Split and Merge Errors}  % abbreviated title (for running head)
%                                     also used for the TOC unless
%                                     \toctitle is used
%
\author{Daniel Haehn\inst{1,2} \and Verena Kaynig-Fittkau\inst{1,2}
\and James Tompkin\inst{1} \and Daniel Berger\inst{2} \and Hanspeter Pfister\inst{1,2} \and Jeff Lichtman\inst{2}}
%
\authorrunning{Daniel Haehn et al.} % abbreviated author list (for running head)
%
%%%% list of authors for the TOC (use if author list has to be modified)
\tocauthor{Daniel Haehn, Verena Kaynig-Fittkau, James Tompkin, and Hanspeter Pfister}
%
\institute{Harvard Paulson School of Engineering and Applied Science, and\\
\and
Harvard Center for Brain Science, Cambridge MA 02138, USA}

\maketitle              % typeset the title of the contribution

\begin{abstract}
Automatic cell image segmentation methods can lead to \emph{split} and \emph{merge} errors, which require correction through proofreading. To aid manual correction of these errors, we develop two classifiers which are able to suggest candidate errors and corrections to the user. These classifiers are informed by training a convolutional neural network with known errors in automatic segmentations by comparison to expert-labeled ground truth. Our network architecture is able to determine potential erroneous regions by considering a wider uncertainty region around an edge. In application, proofreading of electron microscopy image segmentations for connectomics is improved \JT{whatever result we have here}.
\keywords{Segmentation, convolutional neural networks, connectomics.}
\end{abstract}
%

\section{Introduction}

% JT: Trying to write short, as we have only 8 pages.
%\JT{EVERYONE: I am a bit concerned about two reductionist arguments that will be thrown at us: 1. why don't we just try and make the initial segmentation better? 2. at the limit, someone still has to scan the whole volume because your edge error classifier is often wrong. Any ideas?}

%Paragraph one: Provide context to the work.
%What is the task? What is the state of the task?
In connectomics, neuroanatomists build 3D reconstructions of neurons and their connectivity to gain insight into the functional structure of the brain. Rapid progress in automatic sample preparation and electron microscopy (EM) acquisition techniques has made it possible to image large volumes of brain tissue at $\approx6\, nm$ per pixel to identify cells, synapses, and vesicles. For $25\, nm$ thick sections, a $1\, mm^3$ volume of brain contains $10^{15}$ voxels, or 1 petabyte of data: manual annotation is infeasible, and automatic methods are needed \cite{jain2010,Liu2014,GALA2014,kaynig2015large}.

Automatic segmentation and classification of brain tissue is challenging \cite{isbi_challenge}, so learning-based methods are common. The state of the art uses supervised learning with convolutional neural networks \cite{Ciresan:2012f}, or potentially even using unsupervised learning \cite{BogovicHJ13}. Typically, cell membranes are detected in 2D images and the resulting region segmentation is grouped into geometrically-consistent cells across registered sections, or cells are segmented across registered sections in 3D directly. Using dynamic programming techniques \cite{Masci:2013a}, and a GPU cluster, these classifiers can segment  $\approx1$ terabyte of data per hour \cite{kasthuri2015saturated}, which is sufficient to keep up with the 2D data capture process on state-of-the-art electron microscopes (though 3D registration is still an expensive offline operation).
% \cite{Ciresan:2012f,RonnebergerFB15,lee2015recursive}

%State of the art methods use convolutional neural networks to learn cell membranes in 2D images from hand-labeled training data. Then, labeled membranes are grouped into geometrically-consistent cell regions across sections to form 3D image stacks. Using dynamic programming techniques \VKF{cite cdd whole image training paper}, and small GPU clusters \VKF{how many?}, these classifiers can segment about 1 terabyte of data per hour \cite{kasthuri2015saturated,lee2015recursive}, which is the rate necessary to keep up with the data capture process on state-of-the-art electron microscopes. %\VKF{citation for 61 beam microscope?}. \JT{I would skip. There is a Nature press piece (see comment in source), but is there a peer-reviewed academic article? Also, I cited the recent Cell paper for the segmentation rates above; I hope that is ok.} % \url{http://www.nature.com/nature/journal/v503/n7474/full/503147a.html?WT.ec_id=NATURE-20131107}

%Paragraph two: What is the problem in this context? What is the situation that you are trying to correct or overcome?
All automatic methods make errors, and we are left with large data which needs \emph{proofreading} by humans. This crucial task serves two purposes: 1) to correct errors in the segmentation, and 2) to provide large corpora of labeled data to train better automatic segmentation methods. Recent proofreading tools provide intuitive user interfaces to browse segmentation data in 2D and 3D and to identify and manually correct errors \cite{markus_proofreading,raveler,mojo2,haehn_dojo_2014,karimov_guided_volume_editing,uzunbas}. Many kinds of errors exist, such as inaccurate boundaries, but the most common are \emph{split errors}, where a single cell is labeled as two, and \emph{merge errors}, where two cells are labeled as one (Fig.~\ref{fig:merge_error}). With user interaction, split errors can be joined, and the missing boundary in a merge error can be defined with manually-seeded watersheds \cite{haehn_dojo_2014}. However, even with semi-automatic correction tools, the visual inspection of the data to find the errors in the first place takes the majority of the time.

\begin{figure}[t]
\centering
\includegraphics[scale=.1275]{gfx/spliterror_2.png}
\qquad
\includegraphics[scale=.1225]{gfx/mergeerror_new_2.png}
\caption{Split and merge error examples, their corrections, and their ground truths.}
\label{fig:merge_error}
\end{figure}

%Paragraph three: What is the proposed solution at a high level? What is the result of the method, and how does it impact the problem?
Our goal is to add automatic detection of split and merge errors to proofreading tools. Instead of the user visually inspecting the whole data volume carefully to spot errors, we design automatic classifiers that detect split and merge errors in 2D segmentations. Then, a proofreading tool can recommend regions with a high probability of an error to the user, and suggest corrections to accept or reject.

The initial automatic segmentation is constrained by the data rate of the microscope. Given an membrane segmentation from a fast automatic method, our classifiers operate on whole cell regions, which relaxes the constraint on speed: compared to techniques that must analyze every input pixel, this boundary assessment focus reduces the data analysis to the boundaries only, and so allows us to employ wider convolutional neural networks that take regional context and multiple input channels into account. One reason to classify errors on 2D images lies with the cost of 3D registration. This is often slow as it requires non-linear image alignment \cite{akselrod09,Saalfeld2010Asrigidaspossible}. However, typically segmentation results are local decisions at the cell level. In this case, 3D reconstruction is unnecessary and, instead of waiting for the 3D output, proofreading can start immediately to maximize error correction before cell grouping occurs across sections.

%Paragraph five: Optimistically, what is the consequence of the method - what can I do now that I could not do before?
We quantitatively validate our approach with variation of information (VI) versus ground truth expert segmentations. We compare our approach in an experiment against an existing proofreading tool that provides only semi-automatic merge error correction~\cite{haehn_dojo_2014}. Here, with a simulated user, our automatic error suggestions and corrections decrease VI from 0.476 to 0.426, which is in contrast to pure manual or pure automatic methods that can both increase the VI. As a consequence, we are able to provide tools to proofread segmentations more efficiently, and so better tackle large volumes of connectomics imagery.

%\subsection{Contributions}
%
%Given this, we contribute to the literature:
%\begin{enumerate}
%\item One contribution
%\item Two contribution
%\item (Maybe) three contribution
%\end{enumerate}
%
%\begin{figure}
%\missingfigure{Example of a fixed split error. Example of a fixed merge error.}
%\caption{Example of a fixed split error. Example of a fixed merge error.}
%\end{figure}
\section{Related Work}

Paragraph: Work on segmentation in Connectomics...but none of these address specifically learning real edges against edges which lead to split or merge errors.

Arganda-Carreras et al.~\cite{10.3389/fnana.2015.00142} posed the ISBI 2D EM segmentation challenge, where a 30-image corpus of cell `in/out' labels was used to train boundary detection. To overcome the often small amounts of expert-labeled ground-truth segmentation data for training convolutional networks for biomedical tasks, Ronneberger et al.~\cite{RonnebergerFB15} use a contracting/expanding path architecture to enable precise localization.



Paragraph: Work on learning in Connectomics

Paragraph: Most related work. People tried this for 3D! \cite{BogovicHJ13}.

\subsection{Contributions}

Given this, we contribute to the literature:
\begin{enumerate}
\item One contribution
\item Two contribution
\item (Maybe) three contribution
\end{enumerate}
\section{Method}

Paragraph one: Overview of the method. What are the high level steps taken to build the system? What are the key techniques used within the system?

\subsection{Labelled Data - Generating fake splits and merges}

Paragraph: What is the goal of generating labeled data? How do you generate fake splits? How do you generate fake merges? How do you generate patches for training? How can we know that the training data is representative of real data?

\begin{figure}
\missingfigure{Input to learning}
\caption{Show visually each of the inputs to the system on a couple of patches.}
\end{figure}

Using only one binary mask allows us to look at merge errors over more than two regions.


\subsection{Network Design}

Paragraph: What is the rationale behind our network design? Why do we think this will work over other approaches? 

Paragraph: What are the inputs to the design? What is the design? Parameters of the design.

If we have enough space it would be good to have a figure of the network architecture, because we have these 
parallel sub nets that only feed into the mlp together at the very end. 

\begin{figure}[t]
\missingfigure{Network architecture figure}
\caption{Network architecture diagram, showing parallel sub nets that feed into a multi-layer perceptron at the end.}
\end{figure}

\subsection{Training}

Paragraph: How did we train the network? What tricks did we employ (e.g., rotation)? What are the parameters of the training? What software stack did we use? Hardware platform, training time (none very interesting, but for completeness). When do we stop training?

Paragraph: What are the thresholds in the system? How do we pick the thresholds?


\begin{table}
\begin{tabular}{ll}
\toprule
Parameter & Value \\
\midrule
one & \\
two & \\
three & \\
\bottomrule
\end{tabular}
\caption{This is a table of parameters. This is not very interesting, but it's easier to read than in the body text and putting everything together helps the reader quickly assess.}
\end{table}
\section{Evaluation}

Paragraph: introduction of how (broadly) we evaluate the method. Where does the ground truth come from?

Equation: What is VI?

\subsection{Split error evaluation}

Paragraph: What is the process of evaluating split errors?

Paragraph: What do we compare against? What is the result? Why is the performance better?

\begin{table}[t]
\begin{tabular}{ll}
\toprule
Method & VI improvement after fixing split errors \\
\midrule
Jain design & \\
Jain design variation & \\
Our design &  \\
Our design variation & \\
\bottomrule
\end{tabular}
\caption{This is a table of results. It shows the comparison to Jain et al., and the comparison to different variations of these algorithms with the varying overlap regions.}
\label{tab:spliterrorcorrectionperformance}
\end{table}

\subsubsection{Analysis}

Paragraph: Demonstration of ROC curves for VI performance in split error adjustment as the threshold varies.

\begin{figure}[t]
\missingfigure{}
\caption{What does the performance of split error correction look like (ROC curve) as the threshold on edge probability changes?}
\end{figure}

\subsection{Merge error evaluation}

Paragraph: What is the process of evaluating merge errors?

Paragraph: What do we compare against? What is the result? Why is the performance better?

\begin{table}[t]
\begin{tabular}{ll}
\toprule
Method & VI improvement after fixing merge errors \\
\midrule
Our design &  \\
Our design variation & \\
\bottomrule
\end{tabular}
\caption{This is a table of results. It shows our ability to improve VI.}
\end{table}

\begin{figure}[t]
\missingfigure{Examples of fixed edges (one split, one merge) + a failed example of each.}
\caption{One successful split fix, one successful merge fix, and one failed example of each to show limitations.}
\end{figure}
\section{Application - Proof reading}

In our experiments, we observed the best performance using a combination of user guidance and our trained network. In contrast to fully interactive proofreading tools like Dojo, Mojo and Raveler, our system requires only minimal user input. We distinguish between merge and split errors and provide a very simple user interface to correct these (see figure \ref{fig:system}).
The system shows only one potential error - either a false merge or a false split - in the interface. In the case of merge errors, the user is presented the highest scoring five possible boundaries as overlays on the corresponding grayscale image and also a possibility to draw a boundary interactively. The user then chooses one of the suggestions, draws a boundary or marks the cell as correct.  For split errors, the system shows the grayscale image and a possible border and the user marks the cell as correct or indicates he wants to split. Our experiment baseline, the Dojo user study, was limited to 30 minutes and participants performed 59 corrections in average (~30 seconds per correction). We think that even non-experts can perform a correction using our system in 15 seconds and thus, the system increases the proofreading performance.

\begin{figure}
\missingfigure{System of notification within Dojo, for instance.}
\caption{}
\label{fig:system}
\end{figure}
\section{Discussion}




\section{Conclusion}

Paragraph: What we did. What the consequences of what we did are. What is now possible from what we did.

%
% ---- Bibliography ----
%
\bibliographystyle{splncs03}
\bibliography{paper_references}

\end{document}
