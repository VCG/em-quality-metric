\section{Introduction}

%Paragraph one: Provide context to the work.
%What is the task? What is the state of the task?
In the field of connectomics neuroanatomists build 3D reconstructions of neurons and their connectivity to gain insight into the functional structure of the brain. Due to rapid progress in automatic sample preparation and image acquisition techniques in electron microscopy, it is possible to image significant volumes of brain tissue with a resolution high enough to identify synapses and vesicles. With a typical resolution of 6nm per pixel, and a section thickness of 25nm, a $1 mm^3$ volume of brain tissue leads to an image volume of more than $10^15$ voxels or 7 Petabyte of data (todo please check these numbers). Manual annotation of these data sets is unfeasible, and automatic segmentation methods are the main bottleneck in identifying the wiring diagram of the brain. While automatic pipelines have been designed to segment large volumes of brain tissue (cite pipeline paper, cite seungs newest paper, others?), all of these methods need manual proofreading in the end to correct errors in the segmentation.  

%Paragraph two: What is the problem in this context? What is the situation that you are trying to correct or overcome?
A couple of interactive proofreading tools have been developed to facilitate proofreading of automatically generated segmentations (cite mojo, dojo, raveler, what else?). The focus of these tools is to provide the user with an intuitive interface to browse the segmentation data in 2d and 3d (danger I tend to swap 2d and 2D and I don't know which one is right) to identify and manually correct errors. Semi automatic segmentation methods like manually seeded watersheds (cite dojo, or mojo? does raveler do this?) are employed to facility the correction of merge errors, where a boundary was missed by the automatic segmentation and needs to be inserted by the user. IMPORTANT: check if raveler or any other proofreading tool does do automatic detection of possible errors and guides the user if not we have a nice contribution here. 

%Paragraph three: What is the proposed solution at a high level? What is the result of the method, and how does it impact the problem?
Even with semi-automatic proofreading tools, the visual inspection of the data takes considerable time and effort. We designed two classifiers which are trained to automatically detect different types of errors in the segmentation and suggest possible corrections. Instead of having to visually inspect the whole data volume carefully to spot any errors, our classifiers enable the proofreading tool to present the user with regions with a high probability of an error, and in addition suggest possible corrections. This reduces the workload of proofreading to accepting or rejecting solutions proposed by the classifiers. 

%Paragraph four: Justify why this approach is worthwhile. What is the point in trying to learn to classify split or merge edges specifically? Why don't we just try and make the initial segmentation better?
State of the art methods for automatic segmentation of connectomics data uses a convolutional neural network to identify cell membranes in 2d images and then groups the cell regions found to form 3d objects which are geometrically consistent across sections. Using dynamic programming techniques (cite cdd whole image training paper), and small GPU clusters (how many?), these classifiers can segment about 1 Terrabyte of data per hour, which is the rate necessary to keep up with the electron microscope (citation for 61 beam microscope?). Once the initial membrane segmentation has been generated, our classifiers for proofreading operate on the level of whole cell regions. This significantly reduces the data volume and allows us to employ larger networks which take a greater region of context and multiple input channels into account. 

Paragraph five: Optimistically, what is the consequence of the method - what can I do now that I could not do before?

\begin{figure}
\missingfigure{Example of a split error. Example of a merge error.}
\caption{Example of a split error. Example of a merge error.}
\end{figure}