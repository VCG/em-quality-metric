\section{Related Work}

Paragraph: Work on segmentation in Connectomics...but none of these address specifically learning real edges against edges which lead to split or merge errors.

Arganda-Carreras et al.~\cite{10.3389/fnana.2015.00142} posed the ISBI 2D EM segmentation challenge in 2012, where a 30-image corpus of fly cell `in/out' labels was used to train boundary detection. To overcome the often small amounts of expert-labeled ground-truth segmentation data for training convolutional networks for biomedical tasks, Ronneberger et al.~\cite{RonnebergerFB15} use a contracting/expanding path architecture to enable precise localization.

The data we are most concerned with is more difficult than the ISBI 2012 challenge. Our mouse brain data has large intercellular space which is not b

Paragraph: Work on learning in Connectomics

Paragraph: Most related work. People tried this for 3D! \cite{BogovicHJ13}. 

Why don't we do 3D? Mega pipeline - people need to start proof-reading as soon as possible, need to maximize throughput. If we can do a good job on 2D, it eases this process as we can start working before the alignment stages.

\subsection{Contributions}

Given this, we contribute to the literature:
\begin{enumerate}
\item One contribution
\item Two contribution
\item (Maybe) three contribution
\end{enumerate}