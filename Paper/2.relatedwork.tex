\section{Related Work}

Paragraph: Work on segmentation in Connectomics...but none of these address specifically learning real edges against edges which lead to split or merge errors.

Arganda-Carreras et al.~\cite{10.3389/fnana.2015.00142} posed the ISBI 2D EM segmentation challenge in 2012, where a 30-image corpus of fly cell `in/out' labels was used to train boundary detection. To overcome the often small amounts of expert-labeled ground-truth segmentation data for training convolutional networks for biomedical tasks, Ronneberger et al.~\cite{RonnebergerFB15} use a contracting/expanding path architecture to enable precise localization.

The data we are most concerned with is more difficult than the ISBI 2012 challenge. Our mouse brain data has large intercellular space which is not b

Paragraph: Work on learning in Connectomics

Paragraph: Most related work. People tried this for 3D! \cite{BogovicHJ13}. 

\paragraph{Interactive segmentation}

Recent works attack the problem of massive volume segmentation through crowd-sourcing\cite{saalfeld09,anderson2011}. EyeWire~\cite{eyewire2012} asks novice users to participate in a segmentation game for segmenting neuronal structures using a semi-automatic algorithm. D2P ~\cite{Giuly2013DP2} uses a micro-labor workforce approach where local boolean decisions are combined to produce a consensus segmentation. NeuroProof \cite{neuroproof2013} allows interactive learning of agglomeration of over-segmentations of images, based on a random forest classifier. In general, our goal is to correct the output of a segmentation which is thought to be good; hence, our tool would be used after learning a segmentation model to direct user attention to correct likely erroneous areas.

\paragraph{Proofreading Tools}
Recent works in segmentation proofreading have begun to move towards semi-automatic methods. Raveler~\cite{raveler} targets expert users and offers many parameters for tweaking the process at the cost of a higher complexity. Sicat et al.~\cite{markus_proofreading} propose guiding the user to problem areas through a graph abstraction of potential problematic regions. Mojo~\cite{mojo2} and its follow-up Dojo~\cite{haehn_dojo_2014} provide semi-automatic merge error correction, though manual error finding is still required.


Why don't we do 3D? Mega pipeline - people need to start proof-reading as soon as possible, need to maximize throughput. If we can do a good job on 2D, it eases this process as we can start working before the alignment stages.

\subsection{Contributions}

Given this, we contribute to the literature:
\begin{enumerate}
\item One contribution
\item Two contribution
\item (Maybe) three contribution
\end{enumerate}