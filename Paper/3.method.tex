\section{Method}

Paragraph one: Overview of the method. What are the high level steps taken to build the system? What are the key techniques used within the system?

\subsection{Labelled Data - Generating fake splits and merges}

Paragraph: What is the goal of generating labeled data? How do you generate fake splits? How do you generate fake merges? How do you generate patches for training? How can we know that the training data is representative of real data?

\begin{figure}
\missingfigure{Input to learning}
\caption{Show visually each of the inputs to the system on a couple of patches.}
\end{figure}

\subsection{Network Design}

Paragraph: What is the rationale behind our network design? Why do we think this will work over other approaches? 

Paragraph: What are the inputs to the design? What is the design? Parameters of the design.

\subsection{Training}

Paragraph: How did we train the network? What tricks did we employ (e.g., rotation)? What are the parameters of the training? What software stack did we use? Hardware platform, training time (none very interesting, but for completeness). When do we stop training?

Paragraph: What are the thresholds in the system? How do we pick the thresholds?

\begin{table}
\begin{tabular}{ll}
\toprule
Parameter & Value \\
\midrule
one & \\
two & \\
three & \\
\bottomrule
\end{tabular}
\caption{This is a table of parameters. This is not very interesting, but it's easier to read than in the body text and putting everything together helps the reader quickly assess.}
\end{table}